\newpage
\thispagestyle{empty}
\mbox{}

\chapter*{Preface}
This project has been developed by two Software Engineering master students from Aalborg University in the fall of 2012. The report documents the use of the \textit{Safety-Critical-Java} (SCJ) specification draft in terms of developing a use case for it.

\vspace{4mm}
\noindent The report consists of nine chapters excluding the appendix. The first chapter introduces the project scope and is concluded with a problem statement. The next chapter introduces concepts and theory of embedded, real-time and safety-critical systems and is followed by a chapter covering some of the relevant aspects of SCJ. Next, the \textit{Java Optimized Processor} (JOP) is covered, that provides the most significant implementation of SCJ at the moment.

With the theoretical grounds described, the report shifts into a more hands-on approach in terms of covering the development of a library as a use case, more specifically the \textit{Cubesat Space Protocol} (CSP), under SCJ specification restrictions. This will be in the shape of two chapters, one for explaining the CSP protocol and one for the development.

Finally the project is concluded in three closing chapters covering reflection, future work and conclusion. 

\vspace{4mm}
\noindent Source material referenced in this report will be denotated with a number in square brackets that represents an entry in the bibliography. Consulting this entry provides information about the source. For example, \cite{Schoeberl:2012:RepRap}, is an article written by T\'{o}rur Biskopst{\o} Str{\o}m and Martin Schoeberl in 2012 and is about a desktop 3D printer in Safety-Critical Java.

\vspace{5mm}
	\begin{flushright}
\emph{Enjoy reading!} Group sw901e12
	\end{flushright}

\newpage
